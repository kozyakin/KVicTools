% filename: loadhyph-ru.tex
% language: russian
%
% Loader for hyphenation patterns, generated by
%     source/generic/hyph-utf8/generate-pattern-loaders.rb
% See also http://tug.org/tex-hyphen
%
% Copyright 2008-2025 TeX Users Group.
% You may freely use, modify and/or distribute this file.
% (But consider adapting the scripts if you need modifications.)
%
% Once it turns out that more than a simple definition is needed,
% these lines may be moved to a separate file.
%
\begingroup
\input hyphen.tex
\lccode`\-=`\-
% Test for pTeX
\ifx\kanjiskip\undefined
% Test for native UTF-8 (which gets only a single argument)
% That's Tau (as in Taco or ΤΕΧ, Tau-Epsilon-Chi), a 2-byte UTF-8 character
\def\testengine#1#2!{\def\secondarg{#2}}\testengine Τ!\relax
\ifx\secondarg\empty
    % Unicode-aware engine (such as XeTeX or LuaTeX) only sees a single (2-byte) argument
    \message{UTF-8 Russian hyphenation patterns}
    \input hyph-ru.tex
\else
    % 8-bit engine (such as TeX or pdfTeX)
    \message{T2A Russian hyphenation patterns}
    % The old system allows choosing patterns and encodings manually. That mechanism needs to be implemented first in this package, so we still fall back on old system.
    \input enrhm2.tex
    \input ruhyphen.tex
\fi\else
    % pTeX
    \message{T2A Russian hyphenation patterns}
    \input hyph-ru.t2a.tex
\fi
\endgroup
