% !Mode:: "TeX:UTF-8"
% -----------------------------------------------------------------
% Начало преамбулы
%
\documentclass{article}
% Следующий пакет предназначен для установки математических шрифтов,
% совместимых по жирности с TimesNewRoman
\usepackage[varg]{txfonts}

\usepackage[cm-default]{fontspec}
\usepackage{xltxtra}

% Следующий пакет предназначен для совместимости с "русскими"
% файлами использующими кодировку t2a
\usepackage{xecyr}


% Следующий пакет предназначен для стандартной русификации
% элементов оформления
\usepackage[russian]{babel}

\defaultfontfeatures{Mapping=tex-text}
\setmainfont[Numbers=Lowercase]{Times New Roman}
\defaultfontfeatures{Scale=MatchLowercase}
\setsansfont{Arial}
\setmonofont{Courier New}

\setlength{\textwidth}{17cm}
\setlength{\textheight}{24.6cm}
\setlength{\oddsidemargin}{0.5cm}
\setlength{\evensidemargin}{0.5cm}
\setlength{\topmargin}{-1cm}

% Следующие определения \newtheorem... предназначены для
% ПОСЛЕДОВАТЕЛЬНОЙ нумерации теорем, лемм и т.п. в пределах одной главы

\newtheorem{theorem}{Теорема}[section]
\newtheorem{lemma}[theorem]{Лемма}
\newtheorem{definition}[theorem]{Определение}
\newtheorem{property}[theorem]{Утверждение}
\newtheorem{corollary}[theorem]{Следствие}
\newtheorem{example}[theorem]{Пример}
\newtheorem{remark}[theorem]{Замечание}

\newcommand{\proof}{\noindent\textsc{Доказательство.\ }}
\newcommand{\qed}{\ \hfill\hspace*{\fill}
$\vbox{\hrule\hbox{\vrule height1.3ex\hskip1.3ex\vrule}\hrule}$
\hss\vskip\topsep\relax}

%---------------------------------------------
% Обратите внимание на следующие ПЕРЕОПРЕДЕЛЕННЫЕ команды
%
% Следующие строки обновляют нумерацию уравнений в каждой секции (Section)
%
%\makeatletter
%\@addtoreset{equation}{section}
%\makeatother

\renewcommand{\theequation}{\thesection.\arabic{equation}}

%---------------------------------------------
% Начало документа

\begin{document}
\pagestyle{empty}

\section*{Шаблон для системы \XeLaTeX}

В этом файле содержится шаблон для подготовки документов в системе \XeLaTeX.
Для трансляции, добавьте Ваше содержимое и сохраните в файл под каким-нибудь
именем. После этого Вы можете оттранслировать файл в русифицированной версии
WinEdt с помощью
команды\\

\indent \verb|TeX\XeLaTeX.|\\

В результате будет создан документ в формате PDF, который можно просмотреть,
нажав на иконку Adobe Acrobat в панели инструментов WinEdt.

Поддержка TrueType шрифтов в системе  \XeLaTeX{} в значительной степени
опирается на использование пакета \textsf{fontspec}, краткие возможности
которого описаны в следующем разделе. Для более полного ознакомления с
возможностями пакета \textsf{fontspec} и, более общо, --- системы \XeLaTeX{}
см. соответствующую документацию по MiKTeX.

\section*{Основы применения пакета \textsf{fontspec}}

Пакет \textsf{fontspec} предоставляет возможности для автоматического выбора
шрифтов документов \LaTeX{} в клоне \XeTeX{}.
Основные команды:\\

\indent \verb|\fontspec[font features]{font display name}|.\\

Пример:

\begin{center}
  \Large
  \fontspec[
      Colour           = 0000CC,
      Numbers          = OldStyle,
      VerticalPosition = Ordinal,
      Variant          = 2
           ]{Monotype Corsiva}
  Мой первый пример применения шрифта Monotype Corsiva
\end{center}

Шрифт по умолчанию default, sans serif, и typewriter могут быть установлены с
помощью команд \verb|\setmainfont|, \verb|\setsansfont| и
\verb|\setmonofont|, соответственно, как показано в преамбуле. В них
используется тот же самый синтаксис, что и в пакете \verb|\fontspec|.
Допустимы все возможные формы шрифтов:

\begin{center}
  {\itshape наклонный и \scshape капитель\dots}\\
  {\sffamily\bfseries Жирный sans serif и \itshape жирный
  наклонный sans serif\dots}
\end{center}

Со шрифтами, установленными в преамбуле, шрифты в математической моде также
изменяются: $\cos(n\pi)=\pm 1$. В этом документе использован математический
шрифт `Euler' (с помощью декларации пакета \textsf{euler} --- или пакета
\textsf{eulervm}, если использован драйвер |xpdfdvimx|), так как
математические шрифты по умолчанию Computer Modern имеют слишком светлую
прорисовку для стандартных TrueType шрифтов.
\[
\mathcal F(s) = \int^\infty_0 f(t) \exp(- st)\,\mathrm{d}t
\]

В преамбуле Вы могли заметить команду \verb|\defaultfontfeatures|. Эта
команда применяет свои аргументы ко всем последующим декларациям выбора
шрифтов. В нашем случае первый аргумент, \verb|Mapping=tex-text|, активирует
стандартные лигатуры \TeX{} типа \verb|``---''| для ``---''. Второй аргумент
автоматически производит масштабирование шрифтов к одной и той же величине
символа 'x'.

Для более детальной информации см. документацию пакета.\\

Ниже приводится пример применения пакета \textsf{babel} для оформления
библиографии

\begin{thebibliography}{2}
\bibitem{} Козякин В.С.,
    \newblock Структура экстремальных траекторий
    дискретных линейных систем и гипотеза Лагариаса-Ванга о
    конечности,
    \newblock \emph{Информационные процессы,} 6, 4, 327--363,
    2006.
\end{thebibliography}

\end{document}
